\documentclass[a4paper,12pt]{article}
\usepackage{nopageno}
\usepackage{tabularx}
\usepackage{polyglossia}
\setmainlanguage{russian}
\usepackage{fontspec}
\usepackage{geometry}
\usepackage[version=4]{mhchem}
\geometry{left=2cm, right=2cm, top=2.5cm, bottom=2.5cm}
\setmainfont{CMU Serif}
\usepackage{siunitx} % For SI units
\sisetup{
  per-mode=symbol, % Display fractions with a slash (e.g., m/s)
  output-decimal-marker={,} % Use a comma as the decimal separator
}

% Manually define Russian unit names
\DeclareSIUnit{\celsius}{°C} % Celsius remains standard
\DeclareSIUnit{\kelvin}{K}
\DeclareSIUnit{\bar}{бар}
\DeclareSIUnit{\pacsal}{Па}
\DeclareSIUnit{\kpacsal}{кПа}
\DeclareSIUnit{\liter}{л}
\DeclareSIUnit{\mole}{моль}
\DeclareSIUnit{\gram}{г}
\DeclareSIUnit{\kg}{кг}
\DeclareSIUnit{\meter}{м}
\DeclareSIUnit{\second}{с}
\DeclareSIUnit{\moleperliter}{M}
\DeclareSIUnit{\ml}{мл}
\DeclareSIUnit{\grampercm}{г/см$^{3}$}

% Disable indentation globally
\setlength{\parindent}{0pt}

% === Begin document ===
\begin{document}
\begin{minipage}[t]{\textwidth}
\vspace*{0pt} % align with top
% Use \leavevmode to start horizontal mode (forces normal line layout)
\leavevmode Фамилия, имя \hrulefill
\begin{center}
  \LARGE Домашняя работа №2
\end{center}
\end{minipage}

\setlength{\parindent}{15pt}
\section*{Задача 1}
а) Рассчитайте pH \SI{0.1}{\moleperliter} раствора серной кислоты.\par
\vspace{5\baselineskip}
б) Каким станет значение pH этого раствора, если добавить к нему равный объем \SI{0.2}{\moleperliter} раствора NaOH?\par
\vspace{5\baselineskip}
\section*{Задача 2}
Запишите уравнение диссоциации фтороводородной кислоты, подпишите сопряженные кислотно-основные пары.\par
\vspace{0.5\baselineskip}
{\centering\ce{\rule{4em}{0.2pt} + H2O -> \rule{8em}{0.2pt}}\par}
\vspace*{2\baselineskip}
\section*{Задача 3}
Распределите формулы кислот в таблицу в соответствии с их силой: \ce{HNO3}, \ce{HF}, \ce{H2SO4}, \ce{HCl}, \ce{HClO4}, \ce{H2SO3}, \ce{H2CO3}, \ce{H2S}.\par
\vspace{0.5\baselineskip}
\newcolumntype{Y}{>{\centering\arraybackslash}X}
\begin{tabularx}{\textwidth}{|Y|Y|}
\hline
Сильные кислоты & Слабые кислоты\\ 
\hline
 & \\ 
 & \\ 
& \\ 
\hline
\end{tabularx}

\section*{Задача 4}
К \SI{25}{\gram} карбоната кальция добавили \SI{20}{\ml} концентрированной соляной кислоты (массовая доля кислоты \SI{36}{$\%$}, плотность \SI{1.16}{\grampercm}). Рассчитайте объем выделившегося газа (н. у.) и массу не растворившегося остатка.

\newpage
\begin{minipage}[t]{\textwidth}
\vspace*{0pt} % align with top
% Use \leavevmode to start horizontal mode (forces normal line layout)
\leavevmode Фамилия, имя \hrulefill
\begin{center}
  \LARGE Домашняя работа №2
\end{center}
\end{minipage}

\setlength{\parindent}{15pt}
\section*{Задача 1}
а) Рассчитайте pH \SI{0.03}{\moleperliter} раствора cоляной кислоты.\par
\vspace{5\baselineskip}
б) Каким станет значение pH этого раствора, если добавить к нему равный объем \SI{0.03}{\moleperliter} раствора NaOH?\par
\vspace{5\baselineskip}
\section*{Задача 2}
Запишите уравнение диссоциации хлорной кислоты, подпишите сопряженные кислотно-основные пары.\par
\vspace{0.5\baselineskip}
{\centering\ce{\rule{4em}{0.2pt} + H2O -> \rule{8em}{0.2pt}}\par}
\vspace*{2\baselineskip}
\section*{Задача 3}
Распределите формулы кислот в таблицу в соответствии с их силой: \ce{HI}, \ce{HBr}, \ce{HNO2}, \ce{HNO3}, \ce{HF}, \ce{H2SO4}, \ce{H2SO3}, \ce{H2CO3}.\par
\vspace{0.5\baselineskip}
\newcolumntype{Y}{>{\centering\arraybackslash}X}
\begin{tabularx}{\textwidth}{|Y|Y|}
\hline
Сильные кислоты & Слабые кислоты\\ 
\hline
 & \\ 
 & \\ 
 & \\ 
\hline
\end{tabularx}
\section*{Задача 4}
К \SI{5}{\gram} карбоната магния добавили \SI{10}{\ml} азотной кислоты с концентрацией \SI{1}{\moleperliter}, при этом наблюдали выделение газа. Рассчитайте объем, который займет этот газ при температуре \SI{25}{\celsius} и давлении \SI{1}{\bar}.
\end{document}

